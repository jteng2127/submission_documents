\documentclass{homework}

\input{particulars}

\begin{document}

\title{Data Science Seminar Report 03/05 \\ App ,遊戲,與區塊鏈的創業經歷與故事}
\author{\chineseName \masterStudentID}
\date{}
\maketitle

\section*{創業心得}

講者在創業的過程中,從第三次創業開始,他選擇以「籌錢」的方式來承擔風險。他認為如果連投資人都無法說服,那這個產品可能就不值得做。他也強調市場變化的重要性,並舉例說明,在經營 Gamelet(Facebook Flash 遊戲)時,學到應該避開紅海市場。但即使市場競爭激烈,只要市場足夠大,還是有機會獲利,例如做農夫農場雖然有上百個競爭對手,仍然能賺錢。

他的第三次創業是 Cubie Messenger,他發現即使已有許多通訊軟體,只要整個產業正在起飛,那麼產品也能跟著成長。他原本以為加速器能訓練如何尋找用戶與市場,結果發現美國的加速器更重視「如何籌資」,因為他們假設產品本身是可以做出來的。對於募資,他分享了一個心得:「與 100 個人講,10 個人喜歡,2 個人投資就好」,並認為這是一種可以訓練的能力。

然而,創業並非總是一帆風順。他的第三款產品最終失敗,原因是社群軟體的 network effect 一旦建立,其他產品很難打入市場。而在創立 Perpetual Protocol(區塊鏈智能合約)時,他抓住了市場時機,趁著較早入場並在熊市期間發展,最終成功起飛。

\section*{軟體產業的未來與職涯}

講者分析了薪資較高的軟體公司類型,包括跨國企業、特定專長領域的公司、與國際市場連結緊密的公司,以及硬體相關公司。他提到,發展路線的差異主要在於穩定度,通常是硬體>軟體>創業。歐美市場的薪資普遍較高,而日本市場則對台灣人較友善。

關於職涯選擇,他建議提升英文表達能力、考慮留學或培養特殊技能,很推薦大家去國外闖闖。另外也能累積特定專業技能,透過開源專案讓國際企業注意到自己,或選擇 remote work 避免簽證問題。另外,也可以透過外商公司轉調到美國,但競爭激烈。

美國的優勢在於數據多,因為大公司都集中於此,因此他們可以分析更多。隨著 AI 讓軟體產業效率更高,初階工程師的需求減少,也讓跨領域技能變得更受重視。

\section*{AI 公司與市場分析}

講者分析了 AI 公司類型,可以類比於網路崛起時的公司生態:

\begin{itemize}
    \item 核心公司,例如 ChatGPT、Google 等
    \item SAAS 服務,如 AI 應用改善業務流程,甚至 SAAS 本身也 AI 化
    \item 原生功能型公司,因 AI 而產生的新商業模式,目前仍處於早期階段,如當年手機崛起後出現的 Uber
\end{itemize}

他也提供市場分析的方法:回顧過去 5 \~{} 10 年的趨勢,觀察大企業的變化。例如,過去 10 年沒人想到 ETF 能佔交易量 40\%,或是虛擬貨幣的崛起,但隨後 Robinhood 與 Binance 都成功抓住了機會。

最後,談到如何說服投資人,他認為關鍵在於不斷改進自己的故事。而人脈也是成功的重要因素,與朋友一起創業較容易成功,因此平時應該累積人脈,或參與創業活動與開源社群認識更多人。

\end{document}