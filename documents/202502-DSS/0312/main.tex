\documentclass{homework}

\input{particulars}

\begin{document}

\title{Data Science Seminar Report 03/12 \\ 區塊鏈安全與數據分析 \\ 講者:吳家志}
\author{\chineseName \masterStudentID}
\date{}
\maketitle

講師分享了他在區塊鏈安全領域的經歷與見解,讓人對鏈上安全的挑戰與發展有更深的理解。他最早在 2013 年於 360 工作,當時就能發現許多 zero-day 漏洞,甚至吸引了華為、軍火商等單位的關注,這顯示了網路安全在全球範圍內的重要性。隨著 2016 年後大家對安全的重視增加,嚴重漏洞逐漸減少,他開始將目光轉向區塊鏈安全,並發現這是一個全新的攻防戰場。

區塊鏈的安全挑戰與 Web2 不同,Web2 駭客的主要現金流來自勒索,而 Web3 駭客則可以直接竊取私鑰,轉走鏈上資產。最早的智能合約甚至有 integer overflow 問題,導致合約內的資金被意外轉移。隨著區塊鏈安全逐漸成為關鍵議題,安全公司除了透過研究漏洞維持競爭力,也需要靠發表論文、參與 CTF 競賽來建立影響力,確保穩定的現金流。

他也提到 ByBit 駭客事件,駭客通常選擇在半夜攻擊,因此安全團隊必須隨時待命,確保能夠快速應對。另一個案例是 Safe 多簽錢包漏洞,因為前端交易資料缺乏檢查,導致惡意升級,使智能合約失去控制,最終造成 15 億美金的損失,這再次凸顯了簽名驗證與交易審查機制的重要性。

面對這些挑戰,數據分析與 AI 技術在區塊鏈安全中扮演關鍵角色。透過鏈上交易監控、搶跑攻擊交易分析、資產追蹤、地址去匿名化等技術,能有效提升應變能力,縮短感知與發現攻擊的時間。他強調,區塊鏈安全領域需要具有駭客潛力、能深入研究某個主題的人才,這類專業能力將成為未來區塊鏈發展的關鍵。

總結來說,講師的分享讓人看到區塊鏈安全的複雜性,也突顯了這個領域的機會與風險並存。未來,如何透過更先進的技術來強化防禦,確保區塊鏈生態的安全,將會是所有開發者與安全專家的重要課題。

\end{document}