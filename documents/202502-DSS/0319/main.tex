\documentclass{homework}

\input{particulars}

\begin{document}

\title{Data Science Seminar Report 03/19 \\ Human-to-X Spatial Proxemics behind Invisible Wireless Signals \\ 講者:巫芳璟}
\author{\chineseName \masterStudentID}
\date{}
\maketitle

講師提到,無線訊號具有極大的潛力,可以應用於人際關係分析、地理位置追蹤以及行為模式識別。例如,透過無限訊號的特性,可以偵測人與人之間的距離,進一步分析實體的 Leader/Follower 關係。此外,無線訊號也能用來判斷地理位置,提供更精確的定位與環境感知。

然而,講師也強調了使用 Wi-Fi 訊號作為偵測工具所面臨的挑戰。首先,隱私問題是一大考量,特別是當無線訊號可以追蹤設備與人員時,如何確保數據的安全性與匿名性是亟待解決的問題。其次,Wi-Fi 訊號的發送間隔(interval)不固定,使得訊號收集過程存在不確定性。此外,Wi-Fi 訊號難以準確估算人數,因為同一個人可能攜帶多個設備,而不同設備的訊號強度也可能有所差異。

針對 Wi-Fi 訊號的應用,講師舉例說明,在公共運輸環境中,例如公車內,可以透過 Wi-Fi 訊號來估算車內乘客數量。此外,無線訊號還能用於收集大量靜態資訊,例如環境中的設備數量與類型。更進一步,透過不同品牌手機的 Wi-Fi 探測信號統計特徵,甚至可以辨識手機品牌,這些資訊對於市場分析與設備管理具有一定價值。

在行為分析方面,講師介紹了如何透過 時間延遲與餘弦相似度計算來偵測 Leader/Follower 關係。透過觀察不同設備之間的訊號變化,可以判斷某人是否在帶領另一人行動,這對於人流分析或社交網絡研究具有重要意義。此外,不僅僅是無線訊號,光信號及其他無線通訊技術也能用來分析人際關係,進一步提升分析的準確度。

不過,講師也指出,無線訊號本身無法提供空間語意,因此若要進一步預測移動軌跡,必須整合視覺數據進行補充。為了解決這個問題,可以對物件進行編碼與相似度計算,進而匹配人員的移動軌跡。此外,透過多設備的聯合定位,每個設備都可以擁有自己的權重,並透過 迭代方式動態更新權重,提升定位的準確性。

整體而言,這場演講讓我對無線訊號的應用與挑戰 有更深入的理解。無線訊號的應用雖然廣泛,但仍面臨隱私保護、訊號不確定性與空間語意缺失等挑戰。未來,若能將 無線訊號與視覺數據、光訊號等技術結合,將能更準確地分析人際關係與移動軌跡,為智慧城市與行為研究帶來更多可能性。

\end{document}