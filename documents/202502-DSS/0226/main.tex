\documentclass{homework}

\input{particulars}

\begin{document}

\title{Data Science Seminar Report 02/26 \\ 研究力提升 – 專業學術簡報與英檢考試策略}
\author{\chineseName \masterStudentID}
\date{}
\maketitle

講師首先介紹了演講的結構,在講一件事情時,可以遵循開頭、主體、結論的結構,讓對方知道自己想表達的重點。

講師也介紹了 QuillBot,能幫忙把文章重新寫的更自然的AI工具,讓我們可以自己決定文章想要什麼語調

在 speaking practice 方面,講師也提供了很多有用的工具:

mocking bird 可以幫你學會怎麼像 native speaker 的講英文,透過實際的演講片段,我們可以重複一次他說的話,讓機器人幫你評分。

Microsoft reading couch 能讓你貼自己的演講稿上去,或是用系統上已經有的段落,讓你練習朗讀那個文章,就能讓系統幫你評分你的語調、發音,達到更好的演講效果。

Speechnotes 則是能幫你把演講的聲音轉換成文字,這樣做的好處是能確保發音正確。如果你講的口齒清晰,讓系統能辨識出正確的字,代表講的很清楚。而托福也是用類似的機制來做到口說測驗,他們也是用機器來聽考生的口說,如果說的不清楚,機器辨識錯誤,造成上下文不連貫,口說的分數就會變低。

組織想法的方式,講師介紹了 PPF (Past, Present and Future),可以用來在口說的時候組織自己的演講內容。這樣也會有各種動詞三態的變化,讓托福、雅思口說更能給你文法分數,如果只用現在式就沒辦法讓他知道你的文法程度。

第二個方式是黃金五法 Direct answer, Reasons, Comparisons, Supporting examples, and Personal experience,讓自己想講的內容能更完整、有結構化的表達出來。

托福跟雅思的差異,雅思口說是問幾個小題,讓你針對問題回答,比較簡單。托福則是比較整合性的口說,要先讀一篇文章,接著聽一段 Professor 的講話,最後再給你 1 分鐘用自己的話親自講一次。

雅思相對比較簡單,但托福比較適合真的想要用在真實情境上的人。這些資訊讓我對未來如何練習英文有更明確的方向。

\end{document}