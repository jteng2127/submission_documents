\documentclass{homework}

\input{particulars}

\begin{document}

\title{Biostatistics Homework 7}
\author{\chineseName \masterStudentID}
\date{}
\maketitle

\section{}

\subsection{}

A 95\% confidence interval with \( n = 12 \):

\[
\alpha = 1 - 0.95 = 0.05, \quad \frac{\alpha}{2} = 0.025, \quad df = n - 1 = 11
\]

From the T-Table:

\[
t^* = 2.201
\]

\subsection{}

A 99\% confidence interval with \( n = 24 \):

\[
\alpha = 1 - 0.99 = 0.01, \quad \frac{\alpha}{2} = 0.005, \quad df = n - 1 = 23
\]

From the T-Table:

\[
t^* = 2.807
\]

\subsection{}

A 90\% confidence interval with \( n = 100 \):

\[
\alpha = 1 - 0.90 = 0.10, \quad \frac{\alpha}{2} = 0.05, \quad df = n - 1 = 99
\]

From the T-Table:

\[
t^* = 1.660
\]

\section{}

\subsection{}

The standard deviation of the mean length \( \bar{X} \):

\[
\sigma_{\bar{X}} = \frac{\sigma}{\sqrt{n}} = \frac{8}{\sqrt{14}} \approx 2.14 \, \text{mm}
\]

\subsection{}

The probability that the sample mean body length is less than 80 mm:

\[
z = \frac{\bar{X} - \mu}{\sigma_{\bar{X}}} = \frac{80 - 86}{2.14} \approx -2.80
\]

From the standard normal table:

\[
P(Z < -2.80) \approx 0.0026
\]

\subsection{}

95\% confidence interval for the mean body length:

\[
\bar{X} \pm z^* \cdot \sigma_{\bar{X}}
\]

For 95\%, \( z^* = 1.96 \):

\[
91 \pm 1.96 \cdot 2.14
\]

\[
91 \pm 4.19 \implies (86.81, 95.19)
\]

\subsection{}

90\% confidence interval for the mean body length:

\[
\bar{X} \pm z^* \cdot \sigma_{\bar{X}}
\]

For 90\%, \( z^* = 1.645 \):

\[
91 \pm 1.645 \cdot 2.14
\]

\[
91 \pm 3.52 \implies (87.48, 94.52)
\]

\subsection{}

95\% confidence interval for \( \mu \) with unknown \( \sigma \) (using sample standard deviation \( s = 8 \)):

\[
\bar{X} \pm t^* \cdot \frac{s}{\sqrt{n}}
\]

For \( n = 14 \), \( df = 13 \), and 95\%, \( t^* = 2.160 \):

\[
91 \pm 2.160 \cdot \frac{8}{\sqrt{14}}
\]

\[
91 \pm 4.62 \implies (86.38, 95.62)
\]

\section{}

\subsection{}

Rejection rule:

For a one-tailed test (\( H_0: \mu = 20 \), \( H_a: \mu > 20 \)) with \( \alpha = 0.05 \), we reject \( H_0 \) if the test statistic \( t \) exceeds the critical value \( t^* \).

Degrees of freedom:
\[
df = n - 1 = 36 - 1 = 35
\]

From the T-Table for \( \alpha = 0.05 \) and \( df = 35 \):
\[
t^* = 1.6905 (\text{average of } df = 30 \text{ and } df = 40)
\]

Test statistic:
\[
t = \frac{\bar{X} - \mu_0}{s / \sqrt{n}} = \frac{20.9 - 20}{2.9 / \sqrt{36}} = \frac{0.9}{0.4833} \approx 1.86
\]

Since \( t = 1.86 > t^* = 1.690 \), we reject \( H_0 \).

\subsection{}

Compute the p-value:

The p-value is the probability of observing a test statistic \( t \geq 1.86 \) under the null hypothesis. From the T-Table for \( df = 35 \):

\[
P(T > 1.86) \approx 0.035
\]

Conclusion:

Since \( p = 0.035 < \alpha = 0.05 \), we reject \( H_0 \).

\section{}

\subsection{}

Rejection rule:

For a two-tailed test (\( H_0: \mu = 30 \), \( H_a: \mu \neq 30 \)) with \( \alpha = 0.05 \), we reject \( H_0 \) if the test statistic \( z \) falls outside the range \( -z^* \) to \( z^* \).

From the Z-Table for \( \alpha/2 = 0.025 \):
\[
z^* = 1.96
\]

Test statistic:
\[
z = \frac{\bar{X} - \mu_0}{\sigma / \sqrt{n}} = \frac{29.3 - 30}{2.5 / \sqrt{20}} = \frac{-0.7}{0.559} \approx -1.25
\]

\subsection{}

Compute the p-value:

\[
P(z < -1.25) = P(z > 1.25) \approx 0.1057
\]

Two-tailed p-value:
\[
p = 2 \cdot 0.1057 = 0.2114
\]

Conclusion:

Since \( p = 0.2114 > \alpha = 0.05 \), we fail to reject \( H_0 \).

\subsection{}

95\% confidence interval for \( \mu \):

\[
\bar{X} \pm z^* \cdot \frac{\sigma}{\sqrt{n}}
\]

For \( z^* = 1.96 \) and \( \frac{\sigma}{\sqrt{n}} = \frac{2.5}{\sqrt{20}} \approx 0.559 \):

\[
29.3 \pm 2.093 \cdot 0.559
\]

\[
29.3 \pm 1.17 \implies (28.13, 30.47)
\]

The 95\% confidence interval for \( \mu \) is (28.13, 30.47).

\section{}

\subsection{}

First, calculate the sample mean \( \bar{X} \) and the sample standard deviation \( s \):

\[
\bar{X} = \frac{\sum X_i}{n} = \frac{0.95 + 1.02 + 1.01 + 0.98}{4} = 0.99
\]

\[
s = \sqrt{\frac{\sum (X_i - \bar{X})^2}{n-1}} = \sqrt{\frac{(0.95 - 0.99)^2 + (1.02 - 0.99)^2 + (1.01 - 0.99)^2 + (0.98 - 0.99)^2}{3}} \approx 0.0316
\]

Standard error of the mean:
\[
SE = \frac{s}{\sqrt{n}} = \frac{0.0316}{\sqrt{4}} = 0.0158
\]

For \( df = n-1 = 3 \) and 95\% confidence level, from the T-Table:
\[
t^* = 3.182
\]

Confidence interval:
\[
\bar{X} \pm t^* \cdot SE = 0.99 \pm 3.182 \cdot 0.0158
\]

\[
0.99 \pm 0.0503 \implies (0.9397, 1.0403)
\]

The 95\% confidence interval for \( \mu \) is (0.94, 1.04)

\subsection{}

\( H_0: \mu = 1 \) (the scale is accurate)

\( H_a: \mu \neq 1 \) (the scale is not accurate)

Test statistic:
\[
t = \frac{\bar{X} - \mu_0}{SE} = \frac{0.99 - 1}{0.0158} \approx -0.63
\]

Degrees of freedom:
\[
df = n - 1 = 3
\]

Critical value for two-tailed test at \( \alpha = 0.05 \):
\[
t^* = 3.182
\]

Since \( |t| = 0.63 < t^* = 3.182 \), we fail to reject \( H_0 \).

\section{}

\subsection{}

Rejection rule: Reject \( H_0 \) if \( \bar{X} > 26 \).

Given \( \mu = 25 \) under \( H_0 \), the sampling distribution of \( \bar{X} \) is:

\[
\bar{X} \sim N(\mu, \sigma_{\bar{X}})
\]

where:
\[
\sigma_{\bar{X}} = \frac{\sigma}{\sqrt{n}} = \frac{50}{\sqrt{900}} = \frac{50}{30} = 1.67
\]

The test statistic under \( H_0 \) is:
\[
z = \frac{\bar{X} - \mu}{\sigma_{\bar{X}}} = \frac{26 - 25}{1.67} \approx 0.60
\]

From the standard normal table:
\[
P(Z > 0.60) \approx 0.2743
\]

Thus, the probability of a Type I error is:
\[
\alpha = P(Z > 0.60) = 0.2743
\]

\subsection{}

For hypothesis \( \mu = 28 \), the sampling distribution of \( \bar{X} \) is:

\[
\bar{X} \sim N(28, 1.67)
\]

The critical value for rejecting \( H_0 \) is \( \bar{X} > 26 \). The test statistic under \( \mu = 28 \) is:

\[
z = \frac{26 - 28}{1.67} \approx -1.20
\]

From the standard normal table:
\[
P(Z \leq -1.20) = 0.1151
\]

Thus, the probability of a Type II error is:
\[
\beta = P(\bar{X} \leq 26 | \mu = 28) = 0.1151
\]

\subsection{}

For the hypothesis \( \mu = 30 \), the sampling distribution of \( \bar{X} \) is:

\[
\bar{X} \sim N(30, 1.67)
\]

The critical value for rejecting \( H_0 \) is \( \bar{X} > 26 \). The test statistic under \( \mu = 30 \) is:

\[
z = \frac{26 - 30}{1.67} \approx -2.40
\]

From the standard normal table:
\[
P(Z \leq -2.40) = 0.0082
\]

Thus, the probability of a Type II error is:
\[
\beta = P(\bar{X} \leq 26 | \mu = 30) = 0.0082
\]

Power of the test when \( \mu = 30 \):
\[
\text{Power} = 1 - \beta = 1 - 0.0082 = 0.9918
\]

\section{}

Hypotheses:
\[
H_0: \mu_d = 0 \quad \text{(no difference in pain levels before and after hypnotism)}
\]
\[
H_a: \mu_d > 0 \quad \text{(pain levels are lower after hypnotism)}
\]


Calculate the differences:
\[
d_i = \text{Before} - \text{After}
\]

\[
d = [-0.2, 4.1, 1.6, 1.8, 3.2, 2.0, 2.9, 9.6]
\]

Mean of the differences (\( \bar{d} \)):
\[
\bar{d} = \frac{\sum d_i}{n} = \frac{-0.2 + 4.1 + 1.6 + 1.8 + 3.2 + 2.0 + 2.9 + 9.6}{8} \approx 3.125
\]

Standard deviation of the differences (\( s_d \)):
\[
s_d = \sqrt{\frac{\sum (d_i - \bar{d})^2}{n-1}} = 2.9114
\]

Standard error of the mean difference (\( SE_d \)):
\[
SE_d = \frac{s_d}{\sqrt{n}} = 1.029
\]

Test statistic (\( t \)):
\[
t = \frac{\bar{d} - \mu_0}{SE_d} = \frac{3.125 - 0}{1.029} \approx 3.04
\]

Degrees of freedom (\( df \)):
\[
df = n - 1 = 7
\]

Critical value (\( t^* \)) at \( \alpha = 0.05 \) for a one-tailed test with \( df = 7 \):
\[
t^* = 1.895
\]

Decision rule: Reject \( H_0 \) if \( t > t^* \).

Since \( 3.04 > 1.895 \), we reject \( H_0 \) and conclude that pain levels are significantly lower after hypnotism.

\section{}

\subsection{}

Hypotheses:

\[
H_0: \mu_{\text{females}} - \mu_{\text{males}} = 0 \quad \text{(no difference in total cholesterol levels)}
\]
\[
H_a: \mu_{\text{females}} - \mu_{\text{males}} \neq 0 \quad \text{(a difference exists)}
\]

The two-sample t-test assuming equal variances is given by:
\[
t = \frac{\bar{X}_1 - \bar{X}_2}{\sqrt{s_p^2 \left(\frac{1}{n_1} + \frac{1}{n_2}\right)}}
\]

where the pooled variance is:
\[
s_p = \frac{(n_1 - 1)s_1^2 + (n_2 - 1)s_2^2}{n_1 + n_2 - 2} = \frac{(71 - 1)(34.79^2) + (37 - 1)(33.24^2)}{71 + 37 - 2} = 1174.53
\]

Standard error:
\[
SE = \sqrt{s_p^2 \left(\frac{1}{n_1} + \frac{1}{n_2}\right)} = \sqrt{1174.53 \left(\frac{1}{71} + \frac{1}{37}\right)} \approx 6.95
\]

Test statistic:
\[
t = \frac{173.70 - 171.81}{6.95} \approx 0.27
\]

Degrees of freedom:
\[
df = n_1 + n_2 - 2 = 71 + 37 - 2 = 106
\]

P-value from the t-table, for \( t = 0.30 \) and \( df = 106 \):
\[
P \approx 0.76 \quad \text{(two-tailed)}
\]

The 95\% confidence interval for the difference in means is:
\[
(\bar{X}_1 - \bar{X}_2) \pm t^* \cdot SE
\]

where \( t^* = 1.984 \) for 95\% confidence and \( df = 106 \).

\[
(173.70 - 171.81) \pm 1.984 \cdot 6.95 = 1.89 \pm 13.79
\]
\[
CI = (-11.9, 15.68)
\]

The P-value of \( 0.76 \) indicates no significant difference in total cholesterol levels between males and females. The confidence interval also contains 0, supporting this conclusion.

\subsection{}

Hypotheses:

\[
H_0: \mu_{\text{males}} - \mu_{\text{females}} = 0 \quad \text{(LDL levels are not higher in males)}
\]
\[
H_a: \mu_{\text{males}} - \mu_{\text{females}} > 0 \quad \text{(LDL levels are higher in males)}
\]

Welch's t-test (assuming unequal variances) is given by:
\[
t = \frac{\bar{X}_1 - \bar{X}_2}{\sqrt{\frac{s_1^2}{n_1} + \frac{s_2^2}{n_2}}}
\]

Degrees of freedom are approximated using:
\[
df = \frac{\left(\frac{s_1^2}{n_1} + \frac{s_2^2}{n_2}\right)^2}{\frac{\left(\frac{s_1^2}{n_1}\right)^2}{n_1 - 1} + \frac{\left(\frac{s_2^2}{n_2}\right)^2}{n_2 - 1}} = 
\frac{\left(\frac{31.05^2}{37} + \frac{29.78^2}{71}\right)^2}{\frac{\left(\frac{31.05^2}{37}\right)^2}{36} + \frac{\left(\frac{29.78^2}{71}\right)^2}{70}} \approx 70.46
\]

Standard error:
\[
SE = \sqrt{\frac{31.05^2}{37} + \frac{29.78^2}{71}} = \sqrt{26.05 + 12.49} \approx \sqrt{38.54} \approx 6.21
\]

Test statistic:
\[
t = \frac{109.44 - 96.38}{6.21} \approx 2.10
\]

P-value from the t-table, for \( t = 2.10 \) and \( df \approx 70.46 \):
\[
P \approx 0.019 \quad \text{(one-tailed)}
\]

Since \( P = 0.019 < \alpha = 0.05 \), we reject \( H_0 \).

\end{document}