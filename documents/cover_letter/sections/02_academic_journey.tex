我於\chineseBachelorUniversity\chineseBachelorDeparment 取得學士學位,目前就讀於\chineseMasterUniversity\chineseMasterInstitute 碩士一年級。

在大學時期,我在多門課程中皆有卓越表現,如程式設計、演算法、資料結構、軟體工程與進階程式競賽技巧等,成績多在 \bfr{95\textasciitilde 100 分},並且班排名位居前 \bfr{6.2\%},在大四下順利\bfr{提前畢業},期間亦曾獲得書卷獎。我也參與了多次程式競賽,取得 \bfr{ICPC 銀牌}與 \bfr{CPE 7 題}等亮眼成績。我也主導了三個學期的\bfr{程式讀書會},提升同學的程式與專案實作技巧,讓校內 CPE 超過 3 題的學生比例從 \bfr{14.63\% 增加到 61.45\%}。

在實務經驗上,我積極參與各種黑客松比賽,曾獲得台積電校園黑客松的\bfr{第三名}與北區 DSC 聯合黑客松的 \bfr{Angular 技術獎}等獎項。此外,大四期間,我曾在精誠資訊擔任\bfr{全端工程師實習生},設計新專案的技術棧,並架設 K8s 等基礎建設與建立 \bfr{GitOps 流程}。暑假加入台積電的 \bfr{DNA IT 實習},導入 Cypress E2E 測試框架,預計能\bfr{提升測試速度達 3 倍},同時協助撰寫腳本以優化系統中 i18n 多語系鍵值的轉換流程。

進入碩士後,我加入 MAPL 實驗室,研究\bfr{基於深度學習的影像壓縮技術}。從軟體開發轉向研究領域,讓我有機會重新學習並打造新的工作流程,包括\bfr{論文閱讀技巧}、建立\bfr{知識管理系統}及定期開會\bfr{分享技術}的模式。我期待在未來的研究中,能進一步提升自己的能力,並貢獻學術研究成果。

在碩士課程中,我也發現許多先前在軟體開發中習以為常的觀念,未必適用於其他領域。例如我過去習慣將系統容器化(Dockerize),但在高效能運算(HPC)環境中,直接在機器上執行反而更有效率,避免 Docker 帶來的額外開銷。這些新的觀點讓我\bfr{重新審視自己的技術習慣},也期待在接下來的碩士期間,能有更多不同思維的碰撞,加強我的專業能力。
