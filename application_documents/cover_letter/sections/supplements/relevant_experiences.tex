\subsubsection*{台灣積體電路製造公司\quad IT 實習生 \hfill 2024/07 - 2024/08}

\begin{itemize}
    \item 負責項目:協助優化專案開發流程與品質
    \begin{itemize}
        \item 導入 Cypress E2E 測試框架,並整合 pre-commit 檢查與 CI 自動化流程,預計可將\bfr{測試速度提升至原有的 3 倍}。
        \item 優化 i18n 多語系鍵值轉換流程,協助撰寫自動化腳本大幅加速轉換作業,將原本\bfr{需耗時數月的手動轉換過程壓縮至數週內}完成。
        \item 推動更嚴謹的 TypeScript 規範,並設計\bfr{漸進式重構方案},讓專案能逐步過渡,從而提升程式碼的可靠性與可維護性。
    \end{itemize}
\end{itemize}

\subsubsection*{精誠資訊股份有限公司\quad 全端工程師實習生 \hfill 2023/08 - 2024/06}

\begin{itemize}
    \item 負責項目:與團隊開發銀行核心微服務系統。
    \begin{itemize}
        \item 制定系統所使用的技術棧,包括自動測試、CI/CD、API 網關、日誌蒐集工具等等。
        \item 架設 Kubernetes 實現可擴展的\bfr{微服務系統}。
        \item 使用 Tekton、ArgoCD 等工具建置 CI/CD 流程並達成 \bfr{GitOps},提升部署效率。
        % \item 撰寫文件規範,制定團隊開發與建置流程。
    \end{itemize}
\end{itemize}

\subsubsection*{海大資工程式讀書會\quad 主辦人與講師 \hfill 2020/10 - 2022/01}

\begin{itemize}
    \item 主要貢獻:開辦三學期的程式讀書會,提升同學的程式邏輯、演算法與資料結構等等。
    \item 負責項目:編寫教學資料、擔任講師、規劃練習賽與處理行政流程。
    \item CPE 檢定:鼓勵同學應用所學積極參與 CPE。
    \begin{itemize}
        \item 海大資工同學於 2020 年 12 月至 2022 年 3 月的期間,3 題以上的人數比例\bfr{從 14.63\% 逐次提升至 61.45\%}。
    \end{itemize}
\end{itemize}

% 7~0題人數統計
% 2020/10: 0/0/4/6/15/13/15/24
% 2020/12: 0/1/1/3/7/24/33/13
% 2021/03: 0/2/2/8/12/36/9/7
% 2021/10: 3/1/1/10/7/10/9/15
% 2021/12: 1/2/7/14/14/24/16/8
% 2022/03: 1/3/16/16/15/10/15/7

% 前綴和
% 2020/10: 0/0/4/10/25/38/53/77
% 2020/12: 0/1/2/5/12/36/69/82
% 2021/03: 0/2/4/12/24/60/69/76
% 2021/10: 3/4/5/15/22/32/41/56
% 2021/12: 1/3/10/24/38/62/78/86
% 2022/03: 1/4/20/36/51/61/76/83

% 前綴和比例
% 2020/10: 00.00, 00.00, 05.19, 12.99, 32.47, 49.35, 68.83, 100.00
% 2020/12: 00.00, 01.22, 02.44, 06.10, 14.63, 43.90, 84.15, 100.00
% 2021/03: 00.00, 02.63, 05.26, 15.79, 31.58, 78.95, 90.79, 100.00
% 2021/10: 05.36, 07.14, 08.93, 26.79, 39.29, 57.14, 73.21, 100.00
% 2021/12: 01.16, 03.49, 11.63, 27.91, 44.19, 72.09, 90.70, 100.00
% 2022/03: 01.20, 04.82, 24.10, 43.37, 61.45, 73.49, 91.57, 100.00