\subsection*{實務專案}

\subsubsection*{SlideTalker - 頭部畫面融合演講影片 \hfill 2023/02 - 2024/01}
\begin{itemize}
    \item 於「2023資訊智慧創新跨域專題競賽」取得\bfr{優等獎}。
    \item 主要貢獻:提供網頁應用程式讓演講者上傳頭像照片和語音,生成\bfr{頭像模擬演講影片}並嵌入講解影片,提高觀眾的專注度與學習效果。
    \item 系統框架:
    \begin{itemize}
        \item 我負責開發 \bfr{React 前端網頁}與架設 \bfr{Kubernetes 分散式系統}。
        \item 後端以 Rust Rocket 開發 API server,並以 Python Flask 開發 Task server。
    \end{itemize}
    \item 技術採用:
    \begin{itemize}
        \item SadTalker:使用頭像照片、語音,生成頭像模擬演講影片。
        \item OpenAI Whisper:辨識語音並生成 AI 字幕內容。
    \end{itemize}
\end{itemize}

\subsubsection*{TravelTracker - 旅遊紀錄整理工具 \hfill 2023/04 - 2024/01}
\begin{itemize}
    \item 由徒步環島啟發,於「2023 全國大專校院智慧創新暨跨域整合創作競賽」\bfr{晉級決賽}。
    \item 主要貢獻:提供手機 APP,整合旅遊的 GPS 軌跡與多媒體資料,並提供 AI 對話與圖像自動標記技術,簡化旅遊過程中的記錄整理工作。
    \item 系統框架:以 \textbf{Flutter} 框架開發,支援\bfr{多平台建置}。
    \item 架構設計:使用 \bfr{Clean Architecture} 設計:
    \begin{itemize}
        \item 分離 Entity 與 Use Case 等模組層,達成\bfr{依賴反轉}與\bfr{關注點分離}並方便單元測試。
        \item 我負責設計架構並實作 Entity、Use Case 等核心程式碼與各層介面。
        \item 使用 Repository 與 Data Source 層管理資料並與資料庫溝通。
        \item 使用 View Model 層管理 UI 狀態以及雙向綁定資料。
    \end{itemize}
    \item 技術採用:
    \begin{itemize}
        \item 透過 GPS 軌跡推算拍攝地點並顯示照片縮圖,讓資料的找尋更為簡便
        \item OpenAI CLIP:自動標註多媒體資料,提高資料搜尋效率。
        \item ChatGPT:利用提示工程進行 AI 對話,將使用者的對話中的要求轉換成自動化操作,讓使用者能夠更快速的完成任務。
    \end{itemize}
\end{itemize}

% \subsubsection*{Lazer Vs Moiety \hfill 2021/10 - 2022/06}
% \begin{itemize}
%     \item 協助「國立台灣海洋大學食品科學系」製作教育遊戲。
%     \item 主要貢獻:使用 \textbf{Unity} 開發\bfr{跨平台教育遊戲},支援網頁應用程式、Android 及 IOS 平台。
%     \item 系統框架:
%     \begin{itemize}
%         \item 我負責關卡設計及使用 \textbf{C\#} 在 Unity 引擎開發遊戲邏輯。
%         \item 串接 Firebase 作為帳號系統與資料庫,紀錄使用者的遊戲進度。
%     \end{itemize}
% \end{itemize}

\subsection*{課程專案}

\subsubsection*{Java 課程 - GPS 軌跡分析系統 \hfill 2022/09 - 2023/01}
\begin{itemize}
    \item 該專案取得班級成績\bfr{排名第一}。
    \item 主要貢獻:在桌面應用程式進行 GPS 軌跡的編輯、平滑化及熱點分析。
    \item 系統框架:
    \begin{itemize}
        \item 我負責使用 \textbf{Java Spring Boot} 框架開發 \textbf{RESTful API} 後端與建立分析邏輯。
        \item 前端以 Vite 框架開發。
    \end{itemize}
\end{itemize}

\subsubsection*{軟體工程課程 - SCAF開發輔助工具 \hfill 2022/09 - 2023/01}
\begin{itemize}
    \item 該專案取得班級成績\bfr{排名第二}。
    \item 主要貢獻:提供網頁與命令列介面,配合軟體的瀑布式及敏捷式開發模式,針對「需求分析-設計-開發-測試-發布」的流程進行輔助。
    \item 技術採用:
    \begin{itemize}
        \item 我負責使用 \textbf{Golang} 的 CLI 套件開發命令列介面,可互動式的設計使指令使用更直覺。
    \end{itemize}
\end{itemize}

% \subsubsection*{物聯網課程 - 居家電量分析系統 \hfill 2023/02 - 2023/06}
% \begin{itemize}
%     \item 主要貢獻:提供一個可以分析及預測家中各電器用電量的系統。
%     \item 系統架構
%     \begin{itemize}
%         \item 我負責使用 \textbf{Flutter} 開發手機 APP,並訓練 \textbf{LSTM} 模型預測未來的用電量。
%         \item 使用感測器透過 WiFi 傳送數據給後端伺服器,並在手機APP上顯示分析結果。
%     \end{itemize}
% \end{itemize}